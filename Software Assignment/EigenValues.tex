\let\negmedspace\undefined
\let\negthickspace\undefined
\documentclass[journal]{IEEEtran}
\usepackage[a5paper, margin=10mm, onecolumn]{geometry}
%\usepackage{lmodern} % Ensure lmodern is loaded for pdflatex
\usepackage{tfrupee} % Include tfrupee package

\setlength{\headheight}{1cm} % Set the height of the header box
\setlength{\headsep}{0mm}     % Set the distance between the header box and the top of the text

\usepackage{gvv-book}
\usepackage{gvv}
\usepackage{cite}
\usepackage{amsmath,amssymb,amsfonts,amsthm}
\usepackage{algorithmic}
\usepackage{graphicx}
\usepackage{textcomp}
\usepackage{xcolor}
\usepackage{txfonts}
\usepackage{listings}
\usepackage{enumitem}
\usepackage{mathtools}
\usepackage{gensymb}
\usepackage{comment}
\usepackage[breaklinks=true]{hyperref}
\usepackage{tkz-euclide} 
\usepackage{listings}
\usepackage{multicol}
\newcounter{sectioicolsn}
% \usepackage{gvv}                                        
\def\inputGnumericTable{}                                 
\usepackage[latin1]{inputenc}                                
\usepackage{color}                                            
\usepackage{array}                                            
\usepackage{longtable}                                       
\usepackage{calc}                                             
\usepackage{multirow}                                         
\usepackage{hhline}                                           
\usepackage{ifthen}                                           
\usepackage{lscape}
\begin{document}

\bibliographystyle{IEEEtran}
\vspace{3cm}
\title{EigenVales Calculation}
\author{AI24BTECH11033-Tanishq Rajiv Bhujbale}
\maketitle
\bigskip

\renewcommand{\thefigure}{\theenumi}
\renewcommand{\thetable}{\theenumi}

\maketitle
\tableofcontents
\newpage

% 1. Introduction
\section{Introduction}
Eigenvalues play a critical role in numerous scientific and engineering fields, including stability analysis, quantum mechanics, and principal component analysis. This report explores various algorithms for eigenvalue calculation, implements one, and analyzes its performance.



% 2. Problem Statement
\section{Problem Statement}
The task is to implement a program to compute the eigenvalues of a given matrix without relying on external libraries for matrix computations. This report documents the algorithm selection, implementation, and performance analysis.

% 3. Algorithm Selection
\section{Algorithm Selection}
\subsection{Overview of Algorithms}
\begin{enumerate}
    \item Power Iteration:

Complexity: $O(n^2)$ per iteration.

- \textbf{Pros}: Simple to implement, works well for finding the largest eigenvalue. 


- \textbf{Cons}: Slow convergence; ineffective for finding all eigenvalues.

  \item QR Algorithm:

Complexity: $O(n^3)$.

- \textbf{Pros}: Reliable, widely used, and capable of finding all eigenvalues.

- \textbf{Cons}: High computational cost, particularly for dense matrices.

 \item Jacobi Method: 

Complexity: $O(n^3)$.

- \textbf{Pros}: Effective for symmetric matrices, high accuracy.

- \textbf{Cons}: Limited to specific types of matrices; not suitable for large systems.

 \item Inverse Iteration:

 Complexity: $O(n^3)$.

- \textbf{Pros}: Efficient for computing eigenvalues near a specified point.

- \textbf{Cons}: Requires a good initial estimate; sensitivity to singular matrices.

\end{enumerate}
\subsection{Chosen Algorithm}
The \textbf{QR Algorithm} was chosen because: 

- Capable of finding all eigenvalues without requiring prior knowledge or good initial estimates.

- Suitable for matrices that are not extremely sparse.

- Balances accuracy and general applicability.


% 4. Algorithm Description
\section{Algorithm Description}
\subsection{QR Algorithm}
The QR Algorithm iteratively factorizes the matrix $A$ into $Q$ and $R$ such that $A = QR$ and updates $A \leftarrow RQ$. After sufficient iterations, $A$ converges to an upper triangular matrix whose diagonal elements are the eigenvalues.

\subsection{Time Complexity Analysis}

- The time complexity of the QR Algorithm for dense matrices is 
$O(n^3)$ per iteration.

- The number of iterations required depends on convergence properties, often related to the spectral gap.

\subsection{Additional Insights}:

- \textbf{Memory Usage}: Moderate, as it involves storing intermediate matrices 
$Q$ and $R$.

- \textbf{Convergence}: Exponential convergence for matrices with distinct eigenvalues.

- \textbf{Suitability}: Works well for small to medium-sized dense matrices but is computationally expensive for very large systems.


\subsection{Pseudocode}
\begin{verbatim}
Input: Matrix A (n x n)
Output: Eigenvalues of A

1. for k = 1 to max_iter:
2.     Compute QR factorization: A = QR
3.     Update A = RQ
4.     Check convergence
5. return diagonal elements of A
\end{verbatim}

% 5. Implementation
\section{Implementation}
\subsection{Programming Language and Libraries}
The implementation was written in C using only standard libraries. Matrix operations such as multiplication and QR decomposition were implemented manually. Dynamic memory allocation was used to handle matrices of varying sizes.

\subsection{Code Snippets}
\lstset{language=C, basicstyle=\ttfamily\small, breaklines=true, frame=single, captionpos=b}

\begin{lstlisting}[caption={C code}]
#include <stdio.h>
#include <stdlib.h>
#include <math.h>

// Function to allocate memory for a 2D matrix
double** allocateMatrix(int n) {
    double** mat = (double**)malloc(n * sizeof(double*));
    for (int i = 0; i < n; i++) {
        mat[i] = (double*)malloc(n * sizeof(double));
    }
    return mat;
}

// Function to free a 2D matrix
void freeMatrix(double** mat, int n) {
    for (int i = 0; i < n; i++) {
        free(mat[i]);
    }
    free(mat);
}

// Function to multiply two matrices
void multiplyMatrices(double** A, double** B, double** result, int n) {
    for (int i = 0; i < n; i++) {
        for (int j = 0; j < n; j++) {
            result[i][j] = 0.0;
            for (int k = 0; k < n; k++) {
                result[i][j] += A[i][k] * B[k][j];
            }
        }
    }
}

// Function to perform QR decomposition using Gram-Schmidt
void qrDecomposition(double** A, double** Q, double** R, int n) {
    for (int k = 0; k < n; k++) {
        for (int i = 0; i < n; i++) R[k][i] = 0.0;
        for (int i = 0; i < n; i++) Q[i][k] = A[i][k];

        for (int i = 0; i < k; i++) {
            for (int j = 0; j < n; j++) {
                R[i][k] += Q[j][i] * A[j][k];
            }
            for (int j = 0; j < n; j++) {
                Q[j][k] -= R[i][k] * Q[j][i];
            }
        }
        R[k][k] = 0.0;
        for (int i = 0; i < n; i++) R[k][k] += Q[i][k] * Q[i][k];
        R[k][k] = sqrt(R[k][k]);
        for (int i = 0; i < n; i++) Q[i][k] /= R[k][k];
    }
}

// Function to compute eigenvalues using the QR algorithm
void qrAlgorithm(double** A, int n, int maxIter) {
    double** Q = allocateMatrix(n);
    double** R = allocateMatrix(n);
    double** temp = allocateMatrix(n);

    for (int iter = 0; iter < maxIter; iter++) {
        qrDecomposition(A, Q, R, n);
        multiplyMatrices(R, Q, temp, n);

        // Copy temp back into A
        for (int i = 0; i < n; i++) {
            for (int j = 0; j < n; j++) {
                A[i][j] = temp[i][j];
            }
        }
    }

    // Print eigenvalues (diagonal of A)
    printf("Eigenvalues:\n");
    for (int i = 0; i < n; i++) {
        printf("%f\n", A[i][i]);
    }

    // Free allocated memory
    freeMatrix(Q, n);
    freeMatrix(R, n);
    freeMatrix(temp, n);
}

int main() {
    int n, maxIter;

    // Input matrix size
    printf("Enter the size of the matrix (n x n): ");
    scanf("%d", &n);

    // Allocate matrix
    double** A = allocateMatrix(n);

    // Input matrix elements
    printf("Enter the elements of the matrix row by row:\n");
    for (int i = 0; i < n; i++) {
        for (int j = 0; j < n; j++) {
            scanf("%lf", &A[i][j]);
        }
    }

    // Input maximum iterations
    printf("Enter the maximum number of iterations: ");
    scanf("%d", &maxIter);

    // Perform QR algorithm
    qrAlgorithm(A, n, maxIter);

    // Free allocated memory
    freeMatrix(A, n);

    return 0;
}

\end{lstlisting}


% 6. Results and Analysis
\section{Results and Analysis}
\subsection{Test Cases}
- Test Case 1: $A = \myvec{ 4 & 2 \\ 1 & 3 }$

  - Computed Eigenvalues: 5 and 2
  
- Test Case 2: $B = \myvec{ 1 & 0 & 0 & 0 & 0 \\ 0 & 2 & 0 & 0 & 0 \\ 0 & 0 & 3 & 0 & 0 \\ 0 & 0 & 0 & 4 & 0 \\ 0 & 0 & 0 & 0 & 5 }$

- Computed Eigenvalues : 1, 2, 3, 4 and 5

\subsection{Performance Analysis}
- Time Complexity: $O(n^3)$

- Convergence rate: Approximately ...

- Limitations: High computational cost for very large matrices.

% 7. Comparison of Algorithms
\section{Comparison of Algorithms}
\begin{tabular}{|l|c|c|c|}
\hline
Algorithm & Time Complexity & Matrix Suitability & Notes \\
\hline
Power Iteration & $O(n^2)$ per iteration & Dominant eigenvalue & Simple but limited \\
QR Algorithm & $O(n^3)$ & General-purpose & Chosen algorithm \\
Jacobi Method & $O(n^3)$ & Symmetric matrices & Stable but slower \\
Inverse Iteration & $O(n^3)$ & Eigenvalues near shifts & Precise but sensitive \\
\hline
\end{tabular}

% 8. Conclusion
\section{Conclusion}
This report demonstrated the implementation of the QR Algorithm for eigenvalue calculation and analyzed its performance. While efficient for dense matrices, future work could explore optimizations for sparse or large-scale matrices.





    \end{document}
