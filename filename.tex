%iffalse
\let\negmedspace\undefined
\let\negthickspace\undefined
\documentclass[journal,12pt,twocolumn]{IEEEtran}
\usepackage{cite}
\usepackage{amsmath,amssymb,amsfonts,amsthm}
\usepackage{algorithmic}
\usepackage{graphicx}
\usepackage{textcomp}
\usepackage{xcolor}
\usepackage{txfonts}
\usepackage{listings}
\usepackage{enumitem}
\usepackage{mathtools}
\usepackage{gensymb}
\usepackage{comment}
\usepackage[breaklinks=true]{hyperref}
\usepackage{tkz-euclide} 
\usepackage{listings}
\usepackage{gvv}
\usepackage{multicol}
\newcounter{sectioicolsn}
\usepackage{amsmath}
%\def\inputGnumericTable{}

\usepackage[latin1]{inputenc}                                
\usepackage{color}                                            
\usepackage{array}                                            
\usepackage{longtable}                                       
\usepackage{calc}                                             
\usepackage{multirow}                                         
\usepackage{hhline}                                           
\usepackage{ifthen}                                           
\usepackage{lscape}
\usepackage{tabularx}
\usepackage{array}
\usepackage{float}


\newtheorem{theorem}{Theorem}[section]
\newtheorem{problem}{Problem}
\newtheorem{proposition}{Proposition}[section]
\newtheorem{lemma}{Lemma}[section]
\newtheorem{corollary}[theorem]{Corollary}
\newtheorem{example}{Example}[section]
\newtheorem{definition}[problem]{Definition}
\newcommand{\BEQA}{\begin{eqnarray}}
\newcommand{\EEQA}{\end{eqnarray}}
\newcommand{\define}{\stackrel{\triangle}{=}}
\theoremstyle{remark}
\newtheorem{rem}{Remark}

% Marks the beginning of the document
\begin{document}
\bibliographystyle{IEEEtran}
\vspace{3cm}
\title{Complex Numbers}
\author{AI24BTECH11033-Tanishq Rajiv Bhujbale}
\maketitle
\newpage
\bigskip

\renewcommand{\thefigure}{\theenumi}
\renewcommand{\thetable}{\theenumi}

\section{MCQs with One or More than One Correct}

\begin{enumerate}
    \item If $\omega$ is an imaginary cube root of unity, then 
    $(1 + \omega - \omega^2)^7$ equals \hfill (1998-2 Marks)
    
    \begin{enumerate}[label=\alph*.]
    \item $128\omega$    
    \item $128\omega$
    \item $128\omega^2$
    \item $-128\omega^2$
    \end{enumerate}
    
    

   \item The value of the sum 
   $\sum_{n=1}^{13} \brak{i^n + i^{n+1}}$
where  $i = \sqrt{-1}$ , equals:

\hfill (1998 - 2 Marks)

\begin{multicols}{4}
    \begin{enumerate}[label=\alph*.]
    \item $ i $
    \item $ i - 1 $
    \item $ -i $
    \item $ 0 $
    \end{enumerate}
\end{multicols}


    \item If 
    \mydet
    {6i & -3i & 1 \\
    4 & 3i & -1 \\
    20 & 3 & i}
    = x + iy,

then \hfill (1998 - 2 Marks)

    \begin{enumerate}[label=\alph*.]
    \item x=3,y=4
    \item x=1,y=3
    \item x=0,y=4
    \item x=0,y=0
    \end{enumerate}
    

    \item Let $ z_1 $ and $ z_2 $ be two distinct complex numbers and let $ z = (1 - t) z_1 + t z_2 $ for some real number $ t $ with $ 0 < t < 1 $. If $ \text{Arg}(w) $ denotes the principal argument of a non-zero complex number $ w $, then:

\hfill (2010)

\begin{enumerate}[label=\alph*.]
    \item $ \abs{z - z_1} + \abs{z - z_2} = \abs{z_1 - z_2} $
    \item $ \text{Arg}(z - z_1) = \text{Arg}(z - z_2) $
    \item $
    \mydet{
    z - z_1 & \overline{z} - \overline{z_1} \\
    z_2 - z_1 & \overline{z_2} - \overline{z_1}
    }
    $
    \item $\text{Arg}(z - z_1) = \text{Arg}(z_2 - z_1)$
\end{enumerate}

   \item Let $ w = \frac{\sqrt{3} + i}{2} $ and $ P = \cbrak{w^n : n = 1, 2, 3, \ldots } $. Further, let 
$
H_1 = \cbrak{ z \in \mathbb{C} \mid \text{Re}(z) > \frac{1}{2} }
$
and
$
H_2 = \cbrak{ z \in \mathbb{C} \mid \text{Re}(z) < -\frac{1}{2} },
$
where $ \mathbb{C} $ is the set of all complex numbers. If $ z_1 \in H_1 $, $ z_2 \in H_2 $, and $ O $ represents the origin, then the angle $ \angle z_1Oz_2 $ is:

\hfill (JEE Adv. 2013)

\begin{multicols}{4}
    \begin{enumerate}[label=\alph*.]
    \item $ \frac{p}{2} $
    \item $ \frac{p}{6} $
    \item $ \frac{2p}{3} $
    \item $ \frac{5p}{6} $
    \end{enumerate}
\end{multicols}

    \item Let $ a, b \in \mathbb{R} $ and $ a^2 + b^2 \ne 0 $. Suppose
    $
    S = \cbrak{ z \in \mathbb{C} \mid z = \frac{1}{a + ibt}, \, t \ne 0 },
    $
    where $ i = \sqrt{-1} $. If $ z = x + iy $ and $ z \in S $, then $(x, y)$ lies on:

    \hfill (JEE Adv. 2016)

    \begin{enumerate}[label=\alph*.]
    \item the circle with radius $ \frac{1}{2a} $ and center $\brak{\frac{1}{2a}, 0}$ for $ a > 0 $, $ b \ne 0 $.
    \item the circle with radius $ \frac{1}{2a} $ and center $\left(\frac{-1}{2a}, 0\right)$ for $ a < 0 $, $ b \ne 0 $.
    \item the x-axis for $ a \ne 0 $, $ b = 0 $.
    \item the y-axis for $\ a = 0 $, $ b \ne 0 $.
    \end{enumerate}

    \item Let $ a, b, x, $ and $ y $ be real numbers such that $ a - b = 1 $ and $ y \ne 0 $. If the complex number $ z = x + iy $ satisfies 
$
\text{Im}\brak{\frac{az + b}{z + 1}} = y,
$
then which of the following is (are) possible value(s) of $ x $?

\begin{enumerate}[label=\alph*.]
    \item $ -1+\sqrt{1-y^2} $
    \item $ -1-\sqrt{1-y^2} $
    \item $ 1+\sqrt{1+y^2} $
    \item $ 1-\sqrt{1+y^2} $
    \end{enumerate}

    \item For a non-zero complex number $ z $ , let $ \text{arg}(z) $ denote the principal argument with $ -\pi < \text{arg}(z)\ \leq \pi $. Then,which of the following statement(s) is (are) FALSE? 
    
    \hfill (JEE Adv. 2018)

    \begin{enumerate}[label=\alph*.]
    \item $ \text{arg}(-1-i)= \frac{\pi}{4},where i= \sqrt{-1} $
    \item The function $ f: \mathbb{R} \to \left(-\pi, \pi\right] $, defined by $ f(t)=arg(-1+it) $ for all $ t \in \mathbb{R} $, is continuous at all points of $ \mathbb{R} $, where $ i= \sqrt{-1} $
    \item For any two complex numbers $ z_1 $ and $ z_2 $ , $ \text{arg}(\frac{z_1}{z_2})-\text{arg}(z_1)+\text{arg}(z_2) $ is an integer multiple of $ 2\pi $
    \item For any three given distinct complex numbers $ z_1,z_2 $ and $ z_3 $, the locus of the point z satisfying the condition $ \text{arg} \brak{\frac{(z-z_1)(z-z_2)}{(z-z_3)(z_2-z_1)}} =\pi $,lies on a straight line
    \end{enumerate}

    \item Let $ s, t, r $ be non-zero complex numbers and L be the set of solutions  $ z = x + iy $ $ \brak{x,y, \in \mathbb{R},i=\sqrt{-1}} $ of the equation $ sz + t\overline{z} + r = 0 $, where $ \overline{z}=x-iy $.Then, which of the following statements(s) is (are) TRUE?

    \hfill (JEE Adv. 2018)

    \begin{enumerate}[label=\alph*.]
    \item Let L has exactly one element, then $ \abs{s} \ne \abs{t} $
    \item If $ \abs{s} = \abs{t} $,then L has infinitely many elements 
    \item The number of elements in $ L \cap \cbrak{z:|z-1+i|=5} $ is at most 2
    \item If L has more than one element, then L has infinitely many elements 
    \end{enumerate}

    \end{enumerate}

    \section{Subjective Problems}

    \begin{enumerate}
    
    \item Express $ \frac{1}{1-\cos\theta +2i\sin\theta} $ in the form $ x + iy $.

    \hfill (1978)

    \item If $ x = a + b $ , $ y = a\gamma + b\beta $ and $ z = a\beta + b\gamma $ where $ \gamma $ and $ \beta $ are the complex cube roots of unity, show that $ xyz = a^3 + b^3 $

    \hfill (1978)

    \item If $ x + iy =\sqrt{\frac{a + ib}{c + id}} $, prove that $ \brak{x^2 + y^2}^2 $ = $ \frac{a^2 + b^2}{c^2 + d^2} $

    \hfill (1979)

    \item Find real values of $ x $ and $ y $ for which the following equation is satisfied $ \frac{\brak{1 + i}x - 2i}{3 + i} + \frac{\brak{2 - 3i}y + i}{3 - i} = i $

    \hfill (1980)

    \item Let the complex numbers $ z_1 , z_2 and z_3 $ be the vertices of an equilateral triangle. Let $ z_0 $ be the circumcentre of the triangle. Then prove that $ z_1^2 + z_2^2 + z_3^2 = 3z_0^2 $.

    \hfill (1981 - 4 Marks)

    \item Prove that the complex numbers $ z_1,z_2 $ and the origin form an equilateral triangle only if $ z_1^2 + z_2^2 - z_1z_2 = 0 $.

    \hfill (1983 - 3 Marks)
    \end{enumerate} 
    
    \end{document}