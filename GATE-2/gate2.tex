\let\negmedspace\undefined
\let\negthickspace\undefined
\documentclass[journal]{IEEEtran}
\usepackage[a5paper, margin=10mm, onecolumn]{geometry}
\usepackage{lmodern} % Ensure lmodern is loaded for pdflatex
 % Include tfrupee package
\setlength{\headheight}{1cm} % Set the height of the header box
\setlength{\headsep}{0mm}     % Set the distance between the header box and the top of the text
\usepackage{enumitem}
\usepackage{gvv-book}
\usepackage{gvv}
\usepackage{cite}
\usepackage{amsmath,amssymb,amsfonts,amsthm}
\usepackage{algorithmic}
\usepackage{graphicx}
\usepackage{textcomp}
\usepackage{xcolor}
\usepackage{txfonts}
\usepackage{listings}
\usepackage{enumitem}
\usepackage{mathtools}
\usepackage{gensymb}
\usepackage{graphicx}
\usepackage{wrapfig}
\usepackage{comment}
\usepackage[breaklinks=true]{hyperref}
\usepackage{tkz-euclide} 
\usepackage{listings}
% \usepackage{gvv}                                        
\def\inputGnumericTable{}                                 
\usepackage[latin1]{inputenc}                                
\usepackage{color}                                            
\usepackage{array}                                            
\usepackage{longtable}                                       
\usepackage{calc}                                             
\usepackage{multirow}                                         
\usepackage{hhline}                                           
\usepackage{ifthen}                                           
\usepackage{lscape}
\begin{document}
\bibliographystyle{IEEEtran}
\vspace{3cm}
\author{AI24BTECH11033 - Tanishq}
\title{GATE 2}
% \maketitle
% \newpage
% \bigskip
{\let\newpage\relax\maketitle}
\title{2014, MA}
\renewcommand{\thefigure}{\theenumi}
\renewcommand{\thetable}{\theenumi}
\setlength{\intextsep}{10pt} % Space between text and floats
\numberwithin{equation}{enumi}
\numberwithin{figure}{enumi}
\renewcommand{\thetable}{\theenumi}
\begin{enumerate}
  \item Let R be a ring. If R$\sbrak{x}$ is a principal ideal domain, then R is necessarily a
  \begin{enumerate}
      \item Unique Factorization Domain
      \item Principal Ideal Domain
      \item Euclidean Domain
      \item Field
  \end{enumerate}
  \item Consider the group homomorphism $\varphi : M_2\brak{\mathbb{R}} \rightarrow \mathbb{R}$ given by $\varphi \brak{A} = \text{trace}\brak{A}$. The kernel of $\varphi$ is isomorphic to which of the following groups?
\begin{enumerate}
    \item $M_2\brak{\mathbb{R}} / \cbrak{ A \in M_2\brak{\mathbb{R}} : \varphi\brak{A} = 0 }$
    \item $\mathbb{R}^2$
    \item $\mathbb{R}^3$
    \item $GL_2\brak{\mathbb{R}}$
\end{enumerate}
   \item Let $X$ be a set with at least two elements. Let $\tau$ and $\tau'$ be two topologies on $X$ such that $\tau' \neq  \cbrak{\phi , X}$. Which of the following conditions is necessary for the identity function id : $\brak{X , \tau} \rightarrow \brak{X , \tau'}$ to be continuous? 
   \begin{enumerate}
      \item $\tau \subseteq \tau'$
      \item $\tau' \subseteq \tau$
      \item no conditions on $\tau$ and $\tau'$
      \item $\tau \cap \tau' = \cbrak{\phi, X}$
  \end{enumerate}
  \item Let $A \in M_3\brak{\mathbb{R}}$ be such that det$(A-I) = 0$, where $I$ denotes the $3 x 3$ identity matrix. If the trace$\brak{A} = 13$ and det$\brak{A} = 32$, then the sum of squares of the eigenvalues of A is \underline{\hspace{2cm}} 
  \item Let $V$ denote the vector space $C^5\sbrak{a, b}$ over $\mathbb{R}$ and $W = \cbrak{ f \in V : \frac{d^4 f}{dt^4} + 2 \frac{d^2 f}{dt^2} - f = 0}$. Then
  \begin{enumerate}
    \item dim$\brak{V} = \infty$ and dim$\brak{W} = \infty$
    \item dim$\brak{V} = \infty$ and dim$\brak{W} = 4$
    \item dim$\brak{V} = 6$ and dim$\brak{W} = 5$
    \item dim$\brak{V} = 5$ and dim$\brak{W} = 4$
\end{enumerate}
\item Let $V$ be a real inner product space of dimension 10. Let $x, y \in V$ be non-zero vectors such that $\langle x, y \rangle = 0$. Then the dimension of $\cbrak{x}^\perp \cap \cbrak{y}^\perp is \underline{\hspace{2cm}}$
\item Consider the following linear programming problem: Minimize $x_1 + x_2$ Subject to:
\begin{align}
2x_1 + x_2 \geq 8 \\ 2x_1 + 5x_2 \geq 10 \\ x_1, x_2 \geq 0
\end{align}
The optimal value to this problem is \underline{\hspace{2cm}}.
\item Let \begin{align} 
f(x) :=\begin{cases}
-3\pi & \text{if} \ -\pi < x \leq 0 \\
3\pi & \text{if} \ 0 < x < \pi
\end{cases} 
\end{align} 
be a periodic function of period $2\pi$. The coefficient of $\sin 3x$ in the Fourier series expansion of $f\brak{x}$ on the interval $\sbrak{-\pi, \pi}$ is \underline{\hspace{2cm}}. \item For the sequence of functions 
\begin{align}
f_n(x) = \frac{1}{x^2} \brak{\sin\frac{1}{n x}}, \quad x \in \lsbrak{1}, \rbrak{\infty},
\end{align}
consider the following quantities expressed in terms of Lebesgue integrals:
\begin{enumerate}
    \item[I.] $\lim\limits_{n \to \infty} \int_1^\infty f_x\brak{x} \, dx.$
    \item[II.] $\int_1^\infty \lim\limits_{n \to \infty} f_n\brak{x} \, dx.$
\end{enumerate}
Which of the following is \textbf{TRUE}?
\begin{enumerate}
    \item The limit in I does not exist.
    \item The integrand in II is not integrable on $\lsbrak{1}, \rbrak{\infty}$.
    \item Quantities I and II are well-defined, but I $\neq$ II.
    \item Quantities I and II are well-defined and I $=$ II.
\end{enumerate}
\item Which of the following statements about the spaces $l^p$ and $L^p\sbrak{0, 1}$ is \textbf{TRUE}?
\begin{enumerate}
    \item $l^3 \subset l^7$ and $L^6\sbrak{0, 1} \subset L^9\sbrak{0, 1}$
    \item $l^3 \subset l^7$ and $L^9\sbrak{0, 1} \subset L^6\sbrak{0, 1}$
    \item $l^7 \subset l^3$ and $L^6\sbrak{0, 1} \subset L^9\sbrak{0, 1}$
    \item $l^7 \subset l^3$ and $L^9\sbrak{0, 1} \subset L^6\sbrak{0, 1}$
\end{enumerate}
\item The maximun modulus of $e^{z^2}$ on the set S=$\cbrak{ z \in \mathbb{C} : 0 \leq \text{Re}\brak{z} \leq 1, 0 \leq \text{Im}\brak{z} \leq 1}$ is
\begin{enumerate}
    \item $\frac{2}{e}$
    \item $e$
    \item $e+1$
    \item $e^2$
\end{enumerate}
\item Let $d_1, d_2$ and $d_3$ be matrices on a set $X$ with at least two elements. Which of the following is \textbf{NOT} a metric on $X$?
\begin{enumerate}
    \item min$\cbrak{d_1, 2}$
    \item max$\cbrak{d_2, 2}$
    \item $\frac{d_3}{1+d_3}$
    \item $\frac{d_1+d_2+d_3}{3}$
\end{enumerate}
\item Let $\Omega = \cbrak{ z \in \mathbb{C} : \text{Im}\brak{z} > 0 }$ and let $C$ be a smooth curve lying in $\Omega$ with initial point $-1 + 2i$ and final point $1 + 2i$. The value of $\int_C \frac{1 + 2z}{1 + z} \, dz$ is
\begin{enumerate}
    \item $4 - \frac{1}{2} \ln 2 + i \frac{\pi}{4}$
    \item $-4 + \frac{1}{2} \ln 2 + i \frac{\pi}{4}$
    \item $4 + \frac{1}{2} \ln 2 - i \frac{\pi}{4}$
    \item $4 - \frac{1}{2} \ln 2 + i \frac{\pi}{2}$
\end{enumerate}
\end{enumerate}
\bibliography{IEEEabrv,gvv_cite}
\end{document}
