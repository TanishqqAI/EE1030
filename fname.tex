%iffalse
\let\negmedspace\undefined
\let\negthickspace\undefined
\documentclass[journal,12pt,twocolumn]{IEEEtran}
\usepackage{cite}
\usepackage{amsmath,amssymb,amsfonts,amsthm}
\usepackage{algorithmic}
\usepackage{graphicx}
\usepackage{textcomp}
\usepackage{xcolor}
\usepackage{txfonts}
\usepackage{listings}
\usepackage{enumitem}
\usepackage{mathtools}
\usepackage{gensymb}
\usepackage{comment}
\usepackage[breaklinks=true]{hyperref}
\usepackage{tkz-euclide} 
\usepackage{listings}
\usepackage{gvv}
\usepackage{multicol}
\newcounter{sectioicolsn}
\usepackage{amsmath}
%\def\inputGnumericTable{}

\usepackage[latin1]{inputenc}                                
\usepackage{color}                                            
\usepackage{array}                                            
\usepackage{longtable}                                       
\usepackage{calc}                                             
\usepackage{multirow}                                         
\usepackage{hhline}                                           
\usepackage{ifthen}                                           
\usepackage{lscape}
\usepackage{tabularx}
\usepackage{array}
\usepackage{float}


\newtheorem{theorem}{Theorem}[section]
\newtheorem{problem}{Problem}
\newtheorem{proposition}{Proposition}[section]
\newtheorem{lemma}{Lemma}[section]
\newtheorem{corollary}[theorem]{Corollary}
\newtheorem{example}{Example}[section]
\newtheorem{definition}[problem]{Definition}
\newcommand{\BEQA}{\begin{eqnarray}}
\newcommand{\EEQA}{\end{eqnarray}}
\newcommand{\define}{\stackrel{\triangle}{=}}
\theoremstyle{remark}
\newtheorem{rem}{Remark}

% Marks the beginning of the document
\begin{document}
\bibliographystyle{IEEEtran}
\vspace{3cm}
\title{Complex Numbers}
\author{AI24BTECH11033-Tanishq Rajiv Bhujbale}
\maketitle
\newpage
\bigskip

\renewcommand{\thefigure}{\theenumi}
\renewcommand{\thetable}{\theenumi}

\section{MCQs with One or More than One Correct}

\begin{enumerate}
    \item If $\omega$ is an imaginary cube root of unity, then 
    (1+$\omega$-$\omega^2$)$^7$ equals \hfill (1998-2 Marks)
    
    \begin{itemize}
    \item[(a)] \(128\omega\)    
    \item[(b)]\(128\omega\)
    \item[(c)]\(128\omega^2\)
    \item[(d]\(-128\omega^2\)
    \end{itemize}
    

   \item The value of the sum 
\[
\sum_{n=1}^{13} \left(i^n + i^{n+1}\right),
\]
where \( i = \sqrt{-1} \), equals:

\hfill (1998 - 2 Marks)

\begin{multicols}{4}
    \begin{itemize}
    \item[(a)] \( i \)
    \item[(b)] \( i - 1 \)
    \item[(c)] \( -i \)
    \item[(d)] \( 0 \)
    \end{itemize}
\end{multicols}


    \item If 
\[
\begin{vmatrix}
6i & -3i & 1 \\
4 & 3i & -1 \\
20 & 3 & i
\end{vmatrix}
= x + iy,
\]
then \hfill (1998 - 2 Marks)

    \begin{itemize}
    \item[(a)] x=3,y=4
    \item[(b)] x=1,y=3
    \item[(c)] x=0,y=4
    \item[(d)] x=0,y=0
    \end{itemize}
    

    \item Let \( z_1 \) and \( z_2 \) be two distinct complex numbers and let \( z = (1 - t) z_1 + t z_2 \) for some real number \( t \) with \( 0 < t < 1 \). If \(\text{Arg}(w)\) denotes the principal argument of a non-zero complex number \( w \), then:

\hfill (2010)

\begin{itemize}
    \item[(a)] \( |z - z_1| + |z - z_2| = |z_1 - z_2| \)
    \item[(b)] \(\text{Arg}(z - z_1) = \text{Arg}(z - z_2)\)
    \item[(c)] \[
    \begin{vmatrix}
    z - z_1 & \overline{z} - \overline{z_1} \\
    z_2 - z_1 & \overline{z_2} - \overline{z_1}
    \end{vmatrix}
    \]
    \item[(d)] \(\text{Arg}(z - z_1) = \text{Arg}(z_2 - z_1)\)
\end{itemize}

   \item Let \( w = \frac{\sqrt{3} + i}{2} \) and \( P = \{w^n : n = 1, 2, 3, \ldots \} \). Further, let 
\[
H_1 = \left\{ z \in \mathbb{C} \mid \text{Re}(z) > \frac{1}{2} \right\}
\]
and
\[
H_2 = \left\{ z \in \mathbb{C} \mid \text{Re}(z) < -\frac{1}{2} \right\},
\]
where \( \mathbb{C} \) is the set of all complex numbers. If \( z_1 \in H_1 \), \( z_2 \in H_2 \), and \( O \) represents the origin, then the angle \( \angle z_1Oz_2 \) is:

\hfill (JEE Adv. 2013)

\begin{multicols}{4}
    \begin{itemize}
    \item[(a)] \(\frac{p}{2}\)
    \item[(b)] \(\frac{p}{6}\)
    \item[(c)] \(\frac{2p}{3}\)
    \item[(d)] \(\frac{5p}{6}\)
    \end{itemize}
\end{multicols}

    \item Let \( a, b \in \mathbb{R} \) and \( a^2 + b^2 \ne 0 \). Suppose
    \[
    S = \left\{ z \in \mathbb{C} \mid z = \frac{1}{a + ibt}, \, t \ne 0 \right\},
    \]
    where \( i = \sqrt{-1} \). If \( z = x + iy \) and \( z \in S \), then \((x, y)\) lies on:

    \hfill (JEE Adv. 2016)

    \begin{itemize}
    \item[(a)] the circle with radius \( \frac{1}{2a} \) and center \(\left(\frac{1}{2a}, 0\right)\) for \( a > 0 \), \( b \ne 0 \).
    \item[(b)] the circle with radius \( \frac{1}{2a} \) and center \(\left(\frac{-1}{2a}, 0\right)\) for \( a < 0 \), \( b \ne 0 \).
    \item[(c)] the x-axis for \( a \ne 0 \), \( b = 0 \).
    \item[(d)] the y-axis for \( a = 0 \), \( b \ne 0 \).
    \end{itemize}

    \item Let \( a, b, x, \) and \( y \) be real numbers such that \( a - b = 1 \) and \( y \ne 0 \). If the complex number \( z = x + iy \) satisfies 
\[
\text{Im}\left(\frac{az + b}{z + 1}\right) = y,
\]
then which of the following is (are) possible value(s) of \( x \)?

\begin{itemize}
    \item[(a)] \(-1+\sqrt{1-y^2}\)
    \item[(b)] \(-1-\sqrt{1-y^2}\)
    \item[(c)] \(1+\sqrt{1+y^2}\)
    \item[(d)] \(1-\sqrt{1+y^2}\)
    \end{itemize}

    \item For a non-zero complex number z,let \(\text{arg}(z)\) denote the principal argument with \( -\pi < \text{arg}(z)\ \leq \pi\). Then,which of the following statement(s) is (are) FALSE? 
    
    \hfill (JEE Adv. 2018)

    \begin{itemize}
    \item[(a)] \(\text{arg}(-1-i)= \frac{\pi}{4},where i= \sqrt{-1}\)
    \item[(b)] The function \(f: \mathbb{R} \to \left(-\pi, \pi\right) \), defined by \(f(t)=\text{arg}(-1+it)\) for all \( t \in \mathbb{R} \), is continuous at all points of \(\mathbb{R}\), where \(i= \sqrt{-1}\)
    \item[(c)] For any two complex numbers \(z_1\) and \(z_2\) , \( \text{arg}(\frac{z_1}{z_2})-\text{arg}(z_1)+\text{arg}(z_2)\) is an integer multiple of \(2\pi\)
    \item[(d)] For any three given distinct complex numbers \(z_1,z_2\) and \(z_3\), the locus of the point z satisfying the condition \( \text{arg}\left( \frac{(z-z_1)(z-z_2)}{(z-z_3)(z_2-z_1)}\right) =\pi\),lies on a straight line
    \end{itemize}

    \item Let \( s, t, r\) be non-zero complex numbers and L be the set of solutions \( z = x + iy \) \(\left(x,y, \in \mathbb{R},i=\sqrt{-1}\right)\) of the equation \(sz + t\overline{z} + r = 0\), where \( \overline{z}=x-iy\).Then, which of the following statements(s) is (are) TRUE?

    \hfill (JEE Adv. 2018)

    \begin{itemize}
    \item[(a)] Let L has exactly one element, then \(|s| \ne |t|\)
    \item[(b)] If \(|s| = |t|\),then L has infinitely many elements 
    \item[(c)] The number of elements in \(L \cap \left\{z:|z-1+i|=5\right\}\) is at most 2
    \item[(d)] If L has more than one element, then L has infinitely many elements 
    \end{itemize}

    \end{enumerate}

    \section{Subjective Problems}

    \begin{enumerate}
    
    \item Express \(\frac{1}{1-cos\theta +2isin\theta}\) in the form\(x + iy\).

    \hfill (1978)

    \item If \(x = a + b\) , \(y = a\gamma + b\beta\) and \(z = a\beta + b\gamma\) where \(\gamma\) and \(\beta\) are the complex cube roots of unity, show that \(xyz = a^3 + b^3\)

    \hfill (1978)

    \item If \(x + iy =\sqrt{\frac{a + ib}{c + id}}\), prove that \(\left(x^2 + y^2\right)^2\)= \(\frac{a^2 + b^2}{c^2 + d^2}\)

    \hfill (1979)

    \item Find real values of \(x\) and \(y\) for which the following equation is satisfied \(\frac{\left(1 + i\right)x - 2i}{3 + i} + \frac{\left(2 - 3i\right)y + i}{3 - i} = i\)

    \hfill (1980)

    \item Let the complex numbers \(z_1 , z_2 and z_3\) be the vertices of an equilateral triangle. Let \(z_0\) be the circumcentre of the triangle. Then prove that \(z_1^2 + z_2^2 + z_3^2 = 3z_0^2 \).

    \hfill (1981 - 4 Marks)

    \item Prove that the complex numbers \(z_1,z_2\) and the origin form an equilateral triangle only if \(z_1^2 + z_2^2 - z_1z_2 = 0\).

    \hfill (1983 - 3 Marks)
    \end{enumerate} 
    
    \end{document}
